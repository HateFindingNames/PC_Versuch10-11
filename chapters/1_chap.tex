%LTeX: language=de-DE
\chapter{Vorbereitung}
	Untersucht wird das Verhalten eines integrierten Spannungsinverters (i.e. Ladungspumpenwandler) vom Typ \textit{LT1054} des Herstellers
	\textsc{Texas Instruments}. Es wird von der in Kapitel 8.2 \textit{Typical Application} des Datenblattes \cite{LT1054.datasheet.TI}
	angegebenen Grundschaltung ausgegangen. Hierin sind die beiden Widerstände \(R_1\) und \(R_2\) so zu dimensionieren, dass
	sich eine Ausgangsspannung von \(U_{out} = \SI{(-5 \pm 0.1)}{V}\) einstellt.
	\begin{equation}
		R_2 = R_1 \left(\frac{\left| U_{out} \right|}{\frac{U_{ref}}{2}-\SI{40}{mV}} + 1\right)
		\label{eq:r1 und r2 dimnsionieren}
	\end{equation}
	Zur Dimensionierung gibt das Datenblatt unter Kapitel 7.3.1 eine Beispielrechnung ausgehend von einer gewünschten Ausgangsspannung
	von \SI{-5}{V} und \(R_1 = \SI{20}{k\ohm}\) gemäß \cref{eq:r1 und r2 dimnsionieren} an. Mit der internen Referenzspannung von \(U_{ref} = \SI{2,5}{V}\)
	und Wahl der Widerstände aus der E12 Reihe ergibt sich hier rechnerisch ein Wert für \(R_2\) von
	\begin{align}
		R_2 &= \SI{22}{k\ohm}\left( \frac{\left| \SI{-5}{V} \right|}{\frac{\SI{2,5}{V}}{2} - \SI{40}{mV}} + 1 \right) \nonumber \\
			&\approx \SI{112,91}{k\ohm}
			\label{eq:R2 rechnerisch}
	\end{align}
	mit dem nächstliegenden Wert der E12-Reihe von \SI{120}{k\ohm}.\par
	Umstellen von \cref{eq:r1 und r2 dimnsionieren} nach \(U_{out}\) und einsetzen von \(R1\) und \(R_2\) ergibt für diesen Fall
	eine Ausgangsspannung von
	\begin{equation}
		U_{out} = \left( \frac{\SI{2,5}{V}}{2}-\SI{40}{mV} \right) \cdot \left(1 - \frac{\SI{120}{k\ohm}}{\SI{20}{k\ohm}}\right) = \SI{-6,05}{V}
	\end{equation}
	was weit außerhalb der gesuchten Toleranz liegt.\par\medskip
	\Cref{fig:kurvenschaar} zeigt die Ausgangsspannung für weitere Kombinationen des Widerstandspaares unter Berücksichtigung der
	im Datenblatt empfohlenen Minimal- und Maximalwerte von \(R_1 \geq \SI{20}{k\ohm}\) und \(\SI{100}{k\ohm} \leq R_2 \leq \SI{300}{k\ohm}\) (Code zu finden in \cref{lst:python code}).
	\begin{figure}[H]
		\centering
		\includesvg[width=.8\textwidth]{plots/resistances}
		\caption[Kurvenschaar zur Ermittlung passender Widerstandswerte]{Kurvenschaar zur Ermittlung passender Widerstandswerte. Grün hinterlegt ist
		die akzeptable Abweichung der Ausgangsspannung vom Sollwert.}
		\label{fig:kurvenschaar}
	\end{figure}
	Es zeigt sich, dass es für keine Kombination zweier Einzelwerte aus der E12-Reihe eine passende Kombination finden lässt.
	Um dem zu entgegnen wird \(R_2\) durch zwei Einzelwiderstände in Reihe mit \SI{100}{k\ohm} und \SI{12}{k\ohm} ersetzt.
	Hieraus ergibt sich nach obiger Gleichung eine Ausgangsspannung von
	\begin{equation}
		U_{out} = \SI{-4,95}{V}
		\label{eq:ausgangsspannung theoretisch}
	\end{equation}\par\medskip
	%
	Mit den Bauteilwerten zur \textit{Typical Application} die im Abschnitt 8.2 des Datenblattes entnommen werden können
	und den ermittelten Werten für \(R_1, R_2\) lässt sich nach Abschnitt 8.2.2.3 \textit{Output Ripple} die Restwelligkeit
	am Ausgang der Schaltung abschätzen.
	\begin{equation}
		\Delta U = \frac{I_{out}}{2\cdot f \cdot C_{4}} + 2 \cdot I_{out} \cdot R_{ESR}
		\label{eq:output ripple}
	\end{equation}
	In \Cref{eq:output ripple} ist \(f\) die Oszillatorfrequenz, die im Datenblatt mit \SI{25}{kHz} angegeben wird. Mit \SI{100}{\micro F}
	für \(C_4\) und einem Laststrom von \SI{25}{mA} errechnet sich für die Ausgangsspannung ein Spitze-Spitze-Wert von
	\begin{equation}
		\Delta U = \frac{\SI{25}{mA}}{2 \cdot \SI{25}{kHz} \cdot \SI{100}{\micro F}} = \SI{5}{mV}
	\end{equation}
	Besitzt der Kondensator am Ausgang jedoch einen nennenswert hohen Serienwiderstand, so darf der zweite Term in \cref{eq:output ripple}
	nicht mehr vernachlässigt werden. So erhöht sich die Restwelligkeit bei einem Serienwiederstand von \SI{0,05}{\ohm} auf
	\SI{7,5}{mV} was einem Anstieg von \SI{150}{\percent} entspricht.