%ltex: language=de-DE
\chapter{Simulation}
	\Cref{fig:grundschaltung} zeigt die in den Simulationen verwendete Grundschaltung.
	\begin{figure}[h]
		\centering
		\includesvg[inkscapelatex=false, width=.8\textwidth]{LTSpice/grundschaltung}
		\caption[Grundschaltung des \textit{LT1054}]{Grundschaltung des \textit{LT1054} gemäß Kapitel 8.2 \textit{Typical Application} des Datenblattes.}-
		\label{fig:grundschaltung}
	\end{figure}
	\newpage
	\section{Ausgangsspannung}
		Zunächst wird das Einschaltverhalten der Schaltung untersucht. Hierzu wird eine Simulation der Schaltung wie sie in \cref{fig:grundschaltung}
		zu sehen ist durchgeführt.
		\begin{figure}[h]
			\centering
			\includesvg[width=.8\textwidth]{LTSpice/2.1_Ausgangsspannung/ausgangsspannung}
			\caption[Verlauf der Ausgangsspannung nach dem Einschaltmoment]{Verlauf der Ausgangsspannung bis \SI{10}{ms} nach dem Einschaltmoment.}
			\label{fig:Ausgangsspannung nach dem einschalten}
		\end{figure}
		\Cref{fig:Ausgangsspannung nach dem einschalten} zeigt den Verlauf der Ausgangsspannung \(U_{out}\) nach dem Einschalten.
		Es ist zu erkennen, dass sich die erwartete Ausgangsspannung erst etwa \SI{2,6}{ms} nach dem Einschalten einstellt.\par
		Im eingeschwungenen Zustand beträgt die mittlere Spannung \(\bar{U}_{out}\) am Ausgang des \textit{LT1054} \SI{-4,9506}{V}.
		Während der simulierte Wert mit einer Abweichung von \SI{0,6}{mV} als deckungsgleich mit dem in \cref{eq:ausgangsspannung theoretisch} gefundenen rechnerischen
		Wert angenommen werden kann, liegt er auch gut im vom Hersteller angegebenen Intervall für \textit{Regulated output Voltage}
		von \(\SI{-4,7}{V} \leq U_{out} \leq \SI{-5,2}{V}\) (vgl. \cref{fig:electrical characteristics}).\par\medskip
		%
		\begin{figure}[h]
			\centering
			\includesvg[width=.8\textwidth]{LTSpice/2.1_Ausgangsspannung/voltage_loss}
			\caption[Verlauf der Ausgangsspannung bei Abwesenheit von \(R_1\) und \(R_2\)]{Verlauf der Ausgangsspannung bei Abwesenheit von \(R_1\) und \(R_2\). Im eingeschwungenen Zustand beträgt die mittlere
			Spannung am Ausgang des Spannungsinverters etwa \SI{5,9262}{V}. Das Zeitintervall, bis der eingeschwungene Zustand erreicht ist erhöht sich deutlich.}
			\label{fig:ohne regelung}
		\end{figure}
		Nach Kapitel 7.1 des Datenblattes entspricht \(U_{out} \approx -U_{in}\), wenn das Bauteil durch Entfernen der Widerstände \(R_1\) und \(R_1\)
		als ungeregelter Spannungsinverter betrieben wird. Die Differenz aus \(U_{in} - |U_{out}|\) wird als
		\textit{Voltage loss} bezeichnet und mit einem typischen Wert von \SI{0,35}{V} bis \SI{1,1}{V} (maximal \SI{0,55}{V} und \SI{1,6}{V} respektive)
		bei Lastströmen zwischen \SI{10}{mA} und \SI{100}{mA} angegeben.
		Eine Simulation bei Gleichhalten aller sonstiger Parameter -- insbesondere des Laststromes -- zeigt eine mittlere Spannung am Ausgang von \SI{5,9262}{V}.
		Hier ergibt sich ein Spannungsverlust von
		\begin{equation}
			U_{loss} = \SI{7}{V} - \SI{5,9262}{V} = \SI{1,0738}{V}
			\label{eq:voltage loss}
		\end{equation}
		was angesichts eines simulierten Laststromes von \SI{50}{mA} erkennbar über dem typischen Wert, jedoch noch unterhalb des maximalen Wertes liegt.
		% \newpage
	\section{Schaltfrequenz}
		Die Schaltung in \cref{fig:grundschaltung} wird auf Periodendauer bzw. Frequenz des internen Oszillators hin überprüft.
		Hierzu werden die zuvor entfernten Regelungswiderstände wieder hinzugefügt, die Schaltung mit \SI{50}{mA} belastet
		und eine Simulation mit der Anweisung \texttt{.tran 10ms startup} gestartet. Ein Plot des Spannungsverlaufs am
		\texttt{OSC}-Anschluss des Spannungsinverters im Zeitbereich \SI{3}{ms} bis \SI{3,5}{ms} zeigt hier eine Periodendauer von
		\(T = \SI{39,58}{\micro s}\) bzw- eine Frequenz von \(f = T^{-1} \approx \SI{25,26}{Hz}\). Das Datenblatt gibt unter Kapitel 6.5
		\textit{Electrical Characteristics} eine Oszillatorfrequenz von \SI{25 \pm 10}{kHz} an. Der Wert der Simulation liegt somit gut
		an der Angabe des Herstellers für den typischen Wert.
	\section{Verhalten bei Sprüngen der Eingangsspannung}\label{sec:ripple am eingang}
		Unter \textit{Electrical Characteristics} wird unter \textit{Input regulation} ein Wert für \(\Delta U_{out,max}\) von \SI{25}{mV} angegeben.
		Die Prüfbedingungen des Herstellers sahen hierbei Kapazitäten von \(C_1 = \SI{10}{\micro F}, C_3 = \SI{2}{nF}\) und \(C_4 = \SI{100}{\micro F}\) vor.
		Für \(C_1\) und \(C_4\) werden hier aufgrund ihres vergleichsweise geringen Serienwiderstandes Tantal-Varianten empfohlen. In der Simulation werden
		die Kapazitätswerte angepasst und für die betreffenden Kondensatoren die parasitären Serienwiderstände auf 0 gesetzt.
		\begin{figure}[h]
			\centering
			\includesvg[width=.8\textwidth]{LTSpice/2.3_Eingangsripple/Verlauf_ausgang}
			\caption[Verlauf der Ausgangsspannung bei rechteckförmigen Sprüngen der Eingangsspannung]{Verlauf der Ausgangsspannung \(U_{out}\) (unten) bei einem der Eingangsspannungsquelle überlagerten Rechtecksignal von \SI{5}{V} (oben).
			Zu erkennen ist ein den Eingangsseitigen Spannungssprüngen folgende, rechteckförmige Überlagerung der Ausgangsspannung von \(\Delta U_{out} \approx \SI{51,19}{mV}\).}
			\label{fig:verlauf eingangsripple}
		\end{figure}
		Eine Simulation mit den Simulationsanweisungen \texttt{PULSE(0 5 1p 1p 1p 10ms 20ms)} für das überlagerte Rechtecksignal und \texttt{.tran 200ms startup}
		zeigt im Zeitbereich zwischen \SI{5}{ms} und \SI{50}{ms} den in \cref{fig:verlauf eingangsripple} gezeigten Spannungsverlauf. Zu erkennen ist ein
		der Ausgangsspannung überlagertes und der Eingangsspannung folgendes Rechtecksignal. \(\Delta U_{out}\) beträgt hier etwa \SI{51,19}{mV}.
		Es ist an dieser Stelle etwas erstaunlich, dass der Wert beim doppelten des vom Hersteller angegebenen Wertes für das Maximum liegt. Erklärungsversuche
		das idealisierte Verhalten der verwendeten Komponenten betreffend haben das Simulationsergebnis nicht nennenswert geändert.
		% \newpage
	\section{Verhalten bei Lastsprüngen}
		Nun wird das Verhalten der Schaltung rechteckförmigen Lastsprüngen untersucht. Die in der Simulation nachzubildenden Testbedingungen
		sehen Lastsprünge zwischen \SI{10}{mA} und \SI{50}{mA} vor. Alle anderen Werte werden gegenüber der Simulation in \cref{sec:ripple am eingang} gleich gehalten.
		\begin{figure}[h]
			\centering
			\includesvg[width=.8\textwidth]{LTSpice/2.4_Lastspruenge/lastspruenge}
			\caption[Verhalten der Ausgangsspannung bei Lastsprüngen]{Verhalten der Ausgangsspannung bei Lastsprüngen zwischen \SI{10}{mA} und \SI{50}{mA}.}
			\label{fig:lastspruenge}
		\end{figure}
		Der rechteckförmigen Signalform der Ausgangsspannung ist eine Restwelligkeit überlagert. Aus diesem Grunde wurden zwei separate Simulationen
		in den Bereichen durchgeführt, in denen das Signal einen niedrigen bzw. hohen Pegel besitzt. Hier wurde anschließend
		vom Simulationsprogramm der Mittelwert errechnet und ausgegeben. Die Differenz beider Mittelwerte liefert hier, wie in \cref{fig:lastspruenge}
		dokumentiert, ein \(\Delta U_{out}\) von \SI{47,8}{mV}.
		% \newpage
	\section{Innenwiderstand}
		Der Innenwiderstand der Schaltung ist gegeben durch
		\begin{equation}
			R_i = \frac{|\Delta \bar{U}_{out}|}{\Delta I_{load}}
			\label{eq:innenwiderstand}
		\end{equation}
		wobei hier \(\Delta \bar{U}_{out}\) den Unterschied der Ausgangsspannung bei verschiedenen Belastungen bezeichnet.\par
		Simulationen mit Lastströmen von \SI{10}{mA} und \SI{100}{mA} im Zeitbereich zwischen \SI{15}{ms} und \SI{20}{ms} nach dem Einschalten
		ergeben die in \cref{tab:innenwiderstaende} gelisteten Werte.
		\begin{table}[h]
			\caption[Ausgangsspannung in Abhängigkeit des Laststromes mit und ohne Regelungsrückführung]{Ausgangsspannung in Abhängigkeit des Laststromes mit und ohne Regelungsrückführung.}
			\centering
			\begin{tabular}{@{}lrrr@{}}%
				\toprule
				&\(U_{out}\) (\SI{10}{mA})	&\(U_{out}\) (\SI{100}{mA})& Diff\\
				\midrule
				Reguliert		&\SI{-4,9984}{V}&\SI{-4,8914}{V}&\SI{107}{mV}\\
				Unreguliert		&\SI{-6,3868}{V}&\SI{-5,3565}{V}&\SI{1,0303}{V}\\
				\bottomrule
			\end{tabular}
			\label{tab:innenwiderstaende}
		\end{table}\par
		%
		Rechnerisch ergeben sich somit dynamische Innenwiderstände von
		\begin{align}
			R_{i,reg} &= \frac{\SI{4,9984}{V} - \SI{4,8914}{V}}{\SI{90}{mA}} \approx \SI{1,19}{\ohm} \nonumber \\
			\\
			R_{i,unreg} &= \frac{\SI{6,3868}{V} - \SI{5,3565}{V}}{\SI{90}{mA}} \approx \SI{11,45}{\ohm} \nonumber \\
			\label{eq:innenwiderstand gerechnet}
		\end{align}
		Zunächst ist auffällig, dass der Innenwiderstand signifikant ansteigt, wenn der Spannungsinverter unregugliert betrieben wird.
		Dieser Wert liegt jedoch nah an der Angabe des Herstellers unter \textit{Output resistance} mit \SI{10}{\ohm} (typisch) (vgl. \cref{fig:electrical characteristics}).
	\section{Restwelligkeit}\label{sec:restwelligkeit}
		Die Güte der Ausgangsspannung bezüglich ihrer Restwelligkeit wird untersucht, indem der Laststrom auf \SI{25}{mA} festgelegt und
		die Simulationsanweisung dahingehend geändert wird, dass der Verlauf der Ausgangsspannung im eingeschwungenen Zustand dargestellt wird.
		\begin{figure}[h]
			\centering
			\begin{subfigure}[c]{.8\textwidth}
				\includesvg[width=\textwidth]{LTSpice/2.6_Restwelligkeit/restwelligkeit_ideal}
				\caption{\(C_1, C_4\) Ideal}
				\label{subfig:restwelligkeit ideal}
			\end{subfigure}
			\begin{subfigure}[c]{.8\textwidth}
				\includesvg[width=\textwidth]{LTSpice/2.6_Restwelligkeit/restwelligkeit_real}
				\caption{\(C_1, C_4\) Real}
				\label{subfig:restwelligkeit real}
			\end{subfigure}
			\caption{Größe und Form der Restwelligkeit am Ausgang. (a) zeigt den Spannungsverlauf bei simulation mit idealen Kondensatoren. (b)
			zeigt den Verlauf bei Austausch von \(C_1\) gegen einen realen Kondensator des types \textit{UPR1C100MAH} und \(C_4\) gegen einen des types \textit{UPL1A101MPH}}
		\end{figure}
		In \cref{subfig:restwelligkeit ideal} zeigt sich so eine Restwelligkeit von \(\Delta U_{out} \approx \SI{5,4}{mV}\).
		Um ein weniger idealisiertes Bild der Situation zu erhalten wurden für den Verlauf in \cref{subfig:restwelligkeit real}
		die beiden Kondensatoren \(C_1\) und \(C_4\) zu realen Aluminium-Elektrolyt Kondensatoren geändert. Hier wurde sich absichtlich über
		die Empfehlung des Herstellers Tantal-Kondensatoren zu verwenden hinweg gesetzt, um die den Unterschied hervorzuheben.
		In diesem Fall erhöht sich die Restwelligkeit um eine Zehnerpotenz auf \SI{34,3}{mV}. Gleichzeitig aber ist eine deutliche Verzerrung
		des Spannungsverlaufes zu erkennen.\par\medskip
		Es zeigt sich, dass sich der parasitäre Widerstand von \(C_4\) sehr viel stärker sowohl auf die Verzerrung, als auch auf die Restwelligkeit
		der Ausgangsspannung auswirkt. Dies ist zu erwarten, da \(C_4\) hier als Ladungsreservoir dient. Ein Serienwiderstand
		verzögert hier sein Laden was sich letztlich in der gezeigten Wellenform äußert.
	\section{Wirkungsgrad}
		Der Wirkungsgrad der Schaltung ist durch
		\begin{equation}
			\eta = \frac{\bar{P}_{out}}{\bar{P}_{in}}
			\label{eq:wirkungsgrad}
		\end{equation}
		gegeben.\par\medskip
		\(\bar{P}_{in}\) ist hier die von der Eingangsspannungsquelle abgegebene Leistung. \(\bar{P}_{out}\) die von der Last
		aufgenommene Leistung. Grundlage ist hier die Konfiguration aus \cref{sec:restwelligkeit}.\par\medskip
		Es ergibt sich ein Wirkungsgrad von
		\begin{equation}
			\eta = \frac{\SI{124,51}{mW}}{\SI{190,24}{mW}} \approx 0,654
		\end{equation}