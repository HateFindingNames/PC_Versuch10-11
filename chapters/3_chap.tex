%ltex: language=de-DE
\chapter{Versuchsdurchführung}
	Wie eingangs erwähnt ist das Residuum des Aufschlusses der kalorimetrischen Bestimmung Gegenstand der nachgelagerten konduktometrischen Titration.
	Folglich wird bei der Versuchsdurchführung in zwei Schritten vorgegangen.
	\section{Kalorimetrische Bestimmung und Aufschluss des Chlorgehalts der Probe}
		Es soll eine kalorimetrische Bestimmung einer unbekannten Probe eines Paraffinöles durchgeführt werden. Hierzu wurde \SI{688,8}{mg} der Probe
		im Tiegel eingewogen und der Druckbehälter wie in \cref{sec:vorbereitung kalorimetrischer aufschluss} beschrieben beschickt. Am Bedienfeld
		des Kalorimeters wird nun jedoch der Modus \texttt{Aufschluss} ausgewählt was unter anderem dazu führt, dass der Druckbehälter am Ende
		des Verbrennungsvorgangs nicht automatisch entgast wird. Dies ist notwendig um durch konduktometrische Titration in Versuch 10
		auf den Chlorgehalt der Probe schließen zu können.\par
		Wie während der Kalibrierung wird im Boden des Druckbehälters wieder \SI{20}{mL} der vorbereiteten Natronlauge eingefüllt. Die bei der Verbrennung frei werdende
		Menge Chlor wird zum einen Teil in der Natronlauge gebunden und befindet sich zum anderen Teil in der Gasphase am Ende der Messung und im
		Kondensat an den Behälterinnenwänden. Nach abgeschlossener Messung wird der Druckbehälter vorsichtig entnommen und bei geschlossenem
		Drosselventil an die in \cref{fig:kalorimeter aufbau} erkennbare Entlüftungsstation angeschlossen. Das Glasgefäß der Entlüftungsstation wird
		mit der übrigen Menge Natronlauge aufgefüllt, sodass nach Öffnen des Ventils das Gas des Druckbehälters beim Entweichen durch
		die Natronlauge geführt wird.\\
		Wenn der Druckbehälter entspannt ist wird er vorsichtig geöffnet und sowohl das Gemisch im Boden des Behälters, als auch das Kondensat an
		seinen Innenwänden durch Abspülen mit VE-Wasser mit dem Inhalt der Entlüftungsstation in einem Messbecher vereint.
	\section{Titration}
		Durch die Verbrennung des Paraffins in Gegenwart von Sauerstoff befindet sich in der zu titrierenden Absorberlösung neben den Chlorid-Ionen
		auch Carbonat-Ionen.