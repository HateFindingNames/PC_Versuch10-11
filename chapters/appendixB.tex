\chapter{Anhang}
	% \lstset{language=Python}
	\begin{lstlisting}[style=python, caption={Python-Code zur Darstellung der Spannungswerte am Ausgang des \textit{LT1054} bei verschiedenen Kombinationen für die
		Widerstandswerte \(R_1, R_2\).}]
		import matplotlib.pyplot as plt
		from eseries import find_nearest, E12

		def getU(R_one, R_two):
			return 1.21 * (1 - (R_two / R_one))

		vals = [22,27,33,39,47,56,68,82,100]

		newvals = []
		for val in vals:
			newvals.append(find_nearest(E12, val*((5 / 1.21) - 1)))
		newvals

		hsize = 10
		fig, ax = plt.subplots(figsize=(hsize,hsize * (9/16)))

		for val in vals:
			Uout = []
			for i, newval in enumerate(newvals):
				Uout.append(getU(val, newval))
				if i == len(newvals) - 1:
					ax.plot(newvals, Uout, label=(str(val) + "k"))

		ax.axhspan(-4.9, -5.1, color="green", alpha=.1)
		ax.set_xlim(100, 300)
		ax.set_title("Kurvenschaar zur Ermittlung passender Widerstandswerte")
		ax.set_xlabel("$R_2$ / $k\Omega$")
		ax.set_ylabel("$U$ / $V$")
		ax.set_xticks(newvals)
		ax.legend()
		ax.grid()
		plt.show()
	\end{lstlisting}
	\label{lst:python code}