%ltex: language=de-DE
\chapter{Diskussion}
	Der gefundene spezifische Brennwert der Probe liegt nah an dem Literaturwert für den spezifischen Brennwert von Benzol und kann
	somit für plausibel befunden werden. Die erhöhte Viskosität der Probe lässt allerdings auf längerkettige Strukturen schließen, die
	wiederum mit Blick auf die Literatur mit höheren spezifischen Brennwerten einhergehen \cite{Tabellenbuch.chemie.Wachter.2012}. Es ist noch unklar, ob die Chloranteile,
	die mit \(\approx \SI{5}{\percent}\) nicht unerheblich sind, hier ihren Beitrag leisten.\par
	Es ist argumentierbar, dass die Abschätzung der eingebrachten Fremdenergie durch den Funken aber insbesondere durch den Faden
	fehlerbehaftet ist. Doch selbst eine Verdopplung der eingebrachten Fremdenergie könnte die Differenz nicht erklären. Ebenso ist
	ein signifikanter Fehler durch den Aufbau und das Messgerät auszuschließen.\\
	Das Ergebnis zum Chlorgehalt des Paraffins erscheint angesichts von typischen Werten \(> \SI{10}{\percent}\) zumindest plausibel.

	Der Verlauf des Leitwertes der konduktometrischen Titration deckt sich sehr gut mit dem erwarteten Verlauf. Allerdings ist nicht restlos geklärt,
	warum der Verlauf dicht um den Äquivalenzpunkt flacher wird.

	Die gekoppelte Natur der Versuche hat sowohl bei der Durchführung, als auch bei der Auswertung vor neue Herausforderungen gestellt.
	Dennoch war das Konzept sehr Lehrreich und hat bisweilen Spaß bereitet.