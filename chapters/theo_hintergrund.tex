%LTex: language=DE-de
\chapter{Theoretischer Hintergrund}\label{sec:theo hintergrund}
	\section*{Kalorische Brennwertbestimmung}
		In der im Versuch durchgeführten Brennwertbestimmung ist die kalorische Grundgleichung \cref{eq:kalorische grundgleichung}
		von zentraler Bedeutung.

		Sie beschreibt den Zusammenhang zwischen der Wärmemenge \(Q\), die benötigt wird, um ein System mit der Wärmekapazität \(k\) um die
		Temperatur \(\Delta T\) zu erwärmen.
		\begin{equation}
			Q = \Delta T \cdot k
			\label{eq:kalorische grundgleichung}
		\end{equation}

		Die sowohl bezüglich des Brennwertes, als auch des Chlorgehaltes zu bestimmende Probe ist ein mit Chlor verunreinigtes Alkan mit einer
		Kettenlänge von \(C_{10-17}\) und werden gemeinhin noch veraltet Paraffine genannt. In der Industrie werden sie immer noch häufig als
		Schmier- und Kühlmittel eingesetzt, oft aber auch als flammhemmender Zusatzstoff in der Elastomerproduktion und -verarbeitung \cite{chloro.paraffine.short-chain,chloro.paraffine.mid-chain}.
		\begin{figure}[h]
			\centering
			\begin{subfigure}[t]{.6\textwidth}
				\centering
				\includesvg[width=\textwidth]{assets/svg/paraffin_rein_struktur}
				\caption{Exemplarische Strukturformel reinen Paraffins.}
				\label{subfig:paraffin rein}
			\end{subfigure}
			% \hfill
			\par\bigskip
			\begin{subfigure}[b]{.6\textwidth}
				\centering
				\includesvg[width=\textwidth]{assets/svg/paraffin_ckw_struktur}
				\caption{Exemplarische Strukturformel eines mit Chlor verunreinigten Paraffins.}
				\label{subfig:paraffin ckw}
			\end{subfigure}
			\caption[Vergleich der Strukturen chlorierten und unchlorierten Paraffins]{Vergleich der Strukturen chlorierten (\cref{subfig:paraffin ckw}) und unchlorierten (\cref{subfig:paraffin rein}Paraffins.}
		\end{figure}

	\section*{Konduktometrische Titration}
		Während es ebenfalls gängige Praxis ist, Titrationen mittels Farbindikatoren wie bspw. Phenolphtalein durchzuführen um aus der Neutralisation
		saurer bzw. alkalischer Lösungen über die verbrauchte Menge des Titers den Äquivalenzpunkt anzuzeigen, kann auch die Änderung des Leitwertes
		des Titranden verwendet werden.\\
		Die Leitfähigkeit -- üblicherweise angegeben in den Einheiten \SI{}{\siemens\centi\metre\squared\per\mole} -- eines Elektrolyten wird maßgeblich bestimmt durch die Anzahl der
		in Lösung befindlichen Ladungsträger, der Anzahl der Elementarladungen je Ladungsträger (Wertigkeit), ihrer Beweglichkeit und der Geometrie des Elektrolytvolumens zwischen
		(und um) den Elektroden.
		\begin{equation}
			G = \frac{A}{l} \sum_i z_i \mathcal{F}c_i u_i = \frac{A}{l} \sigma
			\label{eq:leitfaehigkeit}
		\end{equation}
		Hierbei bildet \(\frac{A}{l}\) den geometrischen Proportionalitätsfaktor, \(\sigma\) die spezifische Leitfähigkeit, \(z\) die Wertigkeit der Ladungsträger, \(\mathcal{F}\) die \textsc{Faraday}-Konstante,
		\(c\) die Ladungsträgerkonzentration und \(u\) ihre Beweglichkeit im Medium \cite{Job.2021.physikalische.chemie}.\\
		Bei der Titration der Absorberlösung bestehend aus chlorierter Natronlauge (Titrand) durch Silbernitratlösung (Titer) handelt es sich um eine Fällungsreaktion gemäß
		\cref{re:NaCl und AgNO3}. Das sich bildende Silberchlorid ist in wässrigem Milieu schwer löslich und fällt aus, wodurch sich die Konzentration
		der in Lösung befindlichen Chlorid-Kationen reduziert.
		\begin{reaction}
			Na_{(aq)}^{+}Cl_{(aq)}^{-} + Ag_{(aq)}^{+}NO_{3(aq)}^{-} -> AgCl v + Na_{(aq)}^{+} + NO_{3(aq)}^{-} "\label{re:NaCl und AgNO3}"
		\end{reaction}
		Der Verlauf der Titration kann in die Abschnitte vor, im und nach dem Äquivalenzpunkt unterteilt werden:\par
		\begin{figure}[h]
			\centering
			\includesvg[width=.75\textwidth]{assets/svg/leitf_verlauf}
			\caption[Erwarteter Verlauf der Leitfähigkeit]{Erwarteter Verlauf der Leitfähigkeit bei Zugabe von Silbernitrat in Natriumchloridlösung \cite{Job.2021.physikalische.chemie}.}
			\label{fig:erwarteter verlauf der leitfaehigkeit}
		\end{figure}
		Vor dem Äquivalenzpunkt werden durch Fällung Chlorid-Kationen (\(\SI{-7,9\cdot10^{-8}}{\metre\squared\per\volt\second}\)) durch weniger bewegliche Nitrat-Kationen
		(\(\SI{-7,4\cdot10^{-8}}{\metre\squared\per\volt\second}\)) im Verhältnis 1:1 ersetzt. Die Konzentration der Natrium-Anionen (\(\SI{5,2\cdot10^{-8}}{\metre\squared\per\volt\second}\)) in Lösung bleibt gleich und ihr Beitrag
		zur Änderung des Leitwertes kann vernachlässigt werden\footnote{In der Tat befinden sich daneben auch Hydroxid- und Hydronium-Ionen in Lösung. Diese stehen jedoch im
		Gleichgewicht zueinander und ihre Beiträge können ebenfalls vernachlässigt werden.}.\par\medskip
		Im Äquivalenzpunkt wurde gerade jedes \(Cl^-\) durch ein \(NO_3^-\) ersetzt. Die Gesamtleitfähigkeit wird nunmehr nur durch die sich in Lösung befindlichen Natrium-Anionen
		bestimmt.\par\medskip
		Nach dem Äquivalenzpunkt werden durch fortlaufende Zugabe von Silbernitratlösung zusätzliche Ladungsträger in Form von \(Ag^+\) (\(\SI{6,4\cdot10^{-8}}{\metre\squared\per\volt\second}\))
		und \(NO_3^-\) eingebracht, die durch fehlende Reaktionspartner in Lösung verbleiben und letztlich zu einem Anstieg der Gesamtleitfähigkeit führen. Hier ist durch den gemeinsamen
		Beitrag der Silber-Anionen und der Nitrat-Kationen mit einer steileren Flanke zu rechnen.