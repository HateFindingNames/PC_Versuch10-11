%ltex: language=de-DE
\chapter{Versuchsdurchführung}
	Wie eingangs erwähnt ist chlorhaltige Absorberlösung des Aufschlusses der kalorimetrischen Bestimmung Gegenstand der nachgelagerten konduktometrischen Titration.
	Folglich wird bei der Versuchsdurchführung in zwei Schritten vorgegangen.
	\section{Kalorimetrische Bestimmung und Aufschluss des Chlorgehalts der Probe}\label{sec:kalorimeter bestimmung unbekannte probe}
		\subsection*{Materialliste Kalorimetrie}
			\underline{Laborgeräte}
			\begin{itemize}
				\item Kalorimeter C 6000
				\item Analysenwaage
				\item Umwälzkühler IKA RC 2 basic
				\item Pinzette
				\item Entlüftungsstation mit
					\begin{itemize}
						\item Drosselventil
						\item Waschflasche
					\end{itemize}
				\item Vollpipette \SI{10}{\milli\litre}
				\item Messkolben \SI{50}{\milli\litre}
			\end{itemize}

			\underline{Verbrauchsmaterial}
			\begin{itemize}
				\item Baumwollfaden (Zündhilfe)
				\item Benzoesäure (Tablettenform)
				\item Natronlauge \(c(NaOH) \approx \SI{1}{\mole\per\litre}\)
				\item VE-Wasser
			\end{itemize}
		Die in \cref{sec:vorbereitung kalorimetrischer aufschluss} beschriebene Kalibrierung wird im Versuch durch Verbrennung von \(\approx \SI{0,5}{g}\) Benzoesäure
		durchgeführt wobei die Kalibrierparameter

		\begin{addmargin}[8mm]{0pt}
			\texttt{Referenzbrennwert}\\
			\texttt{Probengewicht} und\\
			\texttt{Externe Energie 1}
		\end{addmargin}
		bekannt sein müssen. \texttt{Referenzbrennwert} ergibt sich aus dem spezifischen Brennwert der Benzoesäure mit \(H_{Benz} = \SI{26,461}{\joule\per\gram}\).
		Die in Tablettenform vorgelegte Benzoesäure wird auf der Laborwaage im Tiegel eingewogen und ergibt so den Parameter \texttt{Probengewicht} mit \(m_{Benz} = \SI{(506,8 \pm 1)}{mg}\).
		Ein sich ebenfalls im Tiegel befindlicher Baumwollfaden wird elektrisch entzündet und startet so die Verbrennung. Hierdurch wird allerdings eine weitere Wärmeenergie eingebracht,
		die durch Angabe des letzten Parameters \texttt{Externe Energie 1} intern kompensiert wird. Hier wird \SI{50}{J} hinterlegt. Letztlich kann für die Kalibrierung ausgehend von \cref{eq:kalorische grundgleichung} geschrieben werden:
		\begin{equation}
			Q = H_{Benz} \cdot m_{Benz} + Q_{faden} = \Delta T_{kalib} \cdot k
			\label{eq:grundgleichung fuer kalib umgeschrieben}
		\end{equation}

		mit \(k\) als der einzigen Unbekannten.\par
		Die im Zuge der Vorbereitung gemessene Kalorimeterkonstante \(k\) ergibt sich zu
		\begin{equation}
			k_{mess} = \SI{8025}{\joule\per\kelvin}
			\label{eq:gemessene kalorimeterkonstante}
		\end{equation}
		womit bei nachfolgenden Messungen der spezifische Brennwert \(H_{Probe,0}\) der unbekannten Probe ermittelt werden kann.
		
		Hierzu wurde \SI{688,8}{mg} der Probe
		im Tiegel eingewogen und der Druckbehälter wie in \cref{sec:vorbereitung kalorimetrischer aufschluss} beschrieben beschickt. Am Bedienfeld
		des Kalorimeters wird nun jedoch der Modus \texttt{Aufschluss} ausgewählt was unter anderem dazu führt, dass der Druckbehälter am Ende
		des Verbrennungsvorgangs nicht automatisch entgast wird. Dies ist notwendig um durch konduktometrische Titration in Versuch 10
		auf den Chlorgehalt der Probe schließen zu können.\par
		Wie zur Vorbereitung der Kalibrierung wird im Boden des Druckbehälters \SI{10}{mL} der vorbereiteten Natronlauge eingefüllt. Die bei der Verbrennung frei werdende
		Menge Chlor wird zum einen Teil in der Natronlauge gebunden und befindet sich zum anderen Teil in der Gasphase am Ende der Messung und im
		Kondensat an den Behälterinnenwänden. Nach abgeschlossener Messung wird der Druckbehälter vorsichtig entnommen und bei geschlossenem
		Drosselventil an die in \cref{fig:kalorimeter aufbau} erkennbare Entlüftungsstation angeschlossen. Die Waschflasche der Entlüftungsstation wird
		mit der übrigen Menge Natronlauge aufgefüllt, sodass nach Öffnen des Ventils das Gas des Druckbehälters beim Entweichen durch
		die Natronlauge geführt wird.

		Wenn der Druckbehälter entspannt ist, wird er vorsichtig geöffnet und sowohl das Gemisch im Boden des Behälters, als auch das Kondensat an
		seinen Innenwänden durch Abspülen mit VE-Wasser mit dem Inhalt der Entlüftungsstation in einem Messbecher vereint.
	\section{Titration}\label{sec:titration}
		\subsection*{Materialliste Titration}
			\underline{Laborgeräte}
			\begin{itemize}
				\item Bürette
				\item Magnetrührer
				\item Analysenwaage
				\item Wägetrichter
				\item Leitfähigkeitsmessgerät mit Elektrodenstab
				\item Laborstativ
				\item Messkolben \SI{50}{\milli\litre}
				\item Vollpipette \SI{25}{\milli\litre}
				\item Becherglas \SI{400}{\milli\litre}
				\item Messzylinder \SI{10}{\milli\litre}
			\end{itemize}

			\underline{Verbrauchsmaterial}
			\begin{itemize}
				\item Natriumchlorid
				\item Silbernitratlösung \(c(AgNO_3) \approx \SI{1}{\mole\per\litre}\)
				\item Natronlauge \(c(NaOH) \approx \SI{1}{\mole\per\litre}\)
				\item Salpetersäure \(c(HNO_3) \approx \SI{1}{\mole\per\litre}\)
				\item VE-Wasser
			\end{itemize}
		Durch die Verbrennung des Paraffins in Gegenwart von Sauerstoff befindet sich in der zu titrierenden Absorberlösung neben den Chlorid-Ionen
		auch Carbonat-Ionen, die mit dem Titer zusammen Silbernitrat bilden. Chlorid-Ionen bleiben in Folge ungebunden in der Lösung zurück
		und der Verlauf der Leitfähigkeit lässt keinen Rückschluss auf die Konzentration der Chlorid-Ionen mehr zu.
		\begin{reaction}
			2 Ag^+ + 2 NO3^- + CO3^{2-} + 2 Na^+ -> Ag2CO3 v + 2 NaNO3 "\label{re:silbercarbonat bildung}"
		\end{reaction}

		Carbonat-Ionen befinden sich in wässriger Lösung im dissoziativen Gleichgewicht mit ihren ein- und zweiwertigen Säuren.
		Letztere ihrerseits - Kohlensäure - steht in wässriger Lösung im Gleichgewicht mit Wasser und Kohlenstoffdioxid welches seinerseits
		ausgast.
		\newsavebox\VerticalBalanceArrows
		\sbox\VerticalBalanceArrows{\schemestart
			\arrow{<=>}[-90, 0.8]
		\schemestop
		}
		\newsavebox\HorizontalBalanceArrows
		\sbox\HorizontalBalanceArrows{\schemestart
			\arrow{<=>}[0, 0.8]
		\schemestop
		}
		\begin{reactions}%https://ctan.ebinger.cc/tex-archive/macros/latex/contrib/chemformula/chemformula-manual.pdf
			CO2 ^ + H2O {\usebox\HorizontalBalanceArrows} &H2CO3 "\label{re:CO2}"\\
			&{\usebox\VerticalBalanceArrows}\nonumber\\
			H2O <-[+OH^-] H3O+ + &HCO3^- <=>>[pH$\downarrow$] CO2 ^ + 2 H2O "\label{re:Hydrogencarbonat}"\\
			&{\usebox\VerticalBalanceArrows}\nonumber\\
			2 H+ + &CO3^{2-} "\label{re:carbonat}"
		\end{reactions}
		Um die Carbonat-Ionen auszutreiben kann das Gleichgewicht durch Zugabe weiterer Protonen von \cref{re:carbonat} in Richtung \cref{re:CO2}
		und der rechten Seite von \cref{re:Hydrogencarbonat} verschoben werden. Hier bietet sich Salpetersäure an, da ihr basischer Rest gleich dem Kation des Silbernitrats ist und damit
		keinen Einfluss auf die Messung hat (vgl. \cref{re:salpeter in hydrogencarbonat}).
		\begin{reaction}
			HNO3 + H2O + HCO3^- <=>> CO2 + 2 H2O + NO3^- "\label{re:salpeter in hydrogencarbonat}"
		\end{reaction}
		In der Praxis wird mittels Messzylinder etwa \SI{4}{\milli\litre} 1-molare Salpetersäure zur Lösung gegeben und auf dem Magnetrührer
		etwa 10 Minuten gerührt.
	\section{Titration der Absorberlösung}\label{sec:titration der absorberloesung}
		Bedingt durch die Ansäuerung aus \cref{sec:titration} befindet sich noch eine relativ hohe Konzentration Wassertsoff-Ionen in der Lösung
		wodurch die resultierende Leitfähigkeit maßgeblich bestimmt wird. Vor der Titration wird die Lösung mit einer Menge der unkontaminierten
		Absorberlösung neutralisiert während die Leitfähigkeit überwacht wird. Wenn sie ein Minimum erreicht hat, kann mit der Titration begonnen werden.

		Die Titration erfolgt analog dem in \cref{sec:titerbestimmung} beschrieben verfahren.